\section{Introduction}

Distributed systems are prevalent in society to the extent that billions of people either directly or 
indirectly depend on the correct functioning of a distributed system. From banking applications to social
networks, from large-scale data analytics to online video streaming, from web searches to cryptocurrencies,
most of the successful computing applications of today are powered by distributed systems.
The meteoric rise of cloud computing in the past decade has only increased our dependence on these
distributed systems in our lives.

Tasks like monitoring, root cause analysis, performance comprehension require techniques that cut across component,
system, and machine boundaries to collect, correlate, and integrate data. Distributed Tracing is one such cross-cutting technique
that correlates events across the system to a specific request by propagating a unique context per system with the request.
A trace represents the path of one request through the system and contains information such as the timing of requests, 
the events executed, and the nodes where these events were executed. Moreover, traces can be used
to identify slow requests and understand the difference between request executions. 

In this project, we propose to create a visualization tool that uses data from traces to succinctly represent the structure 
and performance of a distributed system. Our tool will allow users to compare the path taken by a group
of requests to the path taken by other requests. We believe that a visualization tool that can represent the
structure of a group of traces, while also visually encoding relevant performance information, will make
it easier for the developers to triage root causes of performance bugs.