\section{Introduction}

Distributed systems form the backbone of many modern applications. The Google search engine,
YouTube, Facebook, and even the mediocre web app built by your neighbor's prodigy kid touches
a distributed system in some way. These systems form much of the mission critical infrastructure
of modern computing and are comprised of multiple nodes with complex communication patterns. Because
of the importance and complexity of distributed systems, it is crucial for the developers building
them to identify performance bottlenecks and visualize different communication patterns. 

Tracing is a technique that allows developers to better understand distributed systems. A trace represents the
path of one request through the system and contains information such as the timing of requests, 
the events executed, and the nodes where these events were executed. Moreover, traces can be used
to identify slow requests and understand the difference between request executions. In this project, we
propose to create a visualization tool that uses data from traces to succintly represent the structure 
and performance of a distributed system. Our tool will allow users to compare the path taken by a group
of requests to the path taken by other requests. We believe that a visualization tool that can represent the
structure of a trace, while also visually encoding releveant performance information, will make distributed
systems more understandable and debuggable.

Both of us have experience with the problem domain. Vaastav has been working on distributed systems for years
and participated in many different projects in the area. Furthermore, his recent research has investigated
tracing in distributed systems. Matheus also has experience with distributed systems through coursework,
research assistanships and graduate research. Tracing is a new topic for him, but he has found it to be
an interesting area that can be useful for debugging and comprehending the interactions of large networks
of computers.