\section{Proposed Solution}

The events of a trace will be transformed into a node-link graph. We will compare the graphs formed by the events of different traces
by partitioning them into side-by-side views. Because of the limited screen space of a computer monitor, we will allow at most two
graphs to be compared at one time. Even though our tool will only allow two graphs to be compared at once, users will still be able
to compare multiple traces by aggregating traces that go through the exact same nodes into one graph. In other words, traces that have
the same structure can be aggregated into one graph. This aggregation procedure will allow us to make comparisons that involve many
traces while only using two DAGs. The links of the graph will be saturated according to the average time it takes for the event in
the end of the link to complete.

One use-case scenario is a developer that wants to find out why two similar requests have very different completion times. The developer
will select the two traces corresponding to the requests. After the traces are selected, our tool will use the events of these traces
to generate a node-link graph. The graph of each trace will be partitioned into a side-by-side view. Using our tool, the developer may
realize that even though both requests are retrieving similar things (let's suppose they are retrieving the profile picture of a user in 
a social network), one of them forms a bigger graph with many red links. The developer may then realize that the fast request used a cached
version of the image, while the other did not. Knowing the cost of a cache miss in this situation, the developer can optimize his service
accordingly. 

\textbf{ADD ILLUSTRATION OF WHAT INTERFACE WILL LOOK LIKE}