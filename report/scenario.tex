\section{Scenario}

\begin{itemize}
\item \textbf{Scenario 1}: A developer wants to find out why two similar requests have very different 
completion times. The developer will select the two traces corresponding to the requests. After the traces are selected, our tool will
use the events of these traces to generate a view showing the differences between the traces as well as showing the
context and timing of the original requests. This will allow the developer to analyze and identify why 2 similar requests have 
different completion times.

\item \textbf{Scenario 2}: A developer wants to find out why requests on Monday are
slower than the requests on Tuesday. The developer will make 2 different selections - selection of traces from Monday and selection
of traces from Tuesday. Our tool will aggregate the 2 selections into representative graphs and then show the difference of these
two aggregate traces. This will allow the developer to possibly figure out a high-level change between the request execution from
Monday to Tuesday.

\item \textbf{Scenario 3}: A developer wants to analyze why a given trace is anomalous as compared to some of the previous
traces. To do this, the developer will first create a selection of traces that will be aggregated down into 1 trace. The developer
will then select the anomalous trace and create a comparison between the anomalous and the representative trace.
Our tool will show the difference between the 2 traces and allow the developer to figure out why a particular trace is anomalous.

\item \textbf{Scenario 4}: A developer wants to understand the communication load between different services of the system.
Our tool will show the developer an overview graph that shows how often 2 services in the distributed system communicate
with each other. This will allow the developer to figure out how an addition of a new service would increase the load on each
service.

\end{itemize}