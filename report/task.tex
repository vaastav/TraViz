\section{Task}

We have identified a total of 4 tasks that we want to support with our viz tool.
2 of these tasks are the main tasks for the project whereas the other 2 are stretch tasks
which we will only implement if we have already completed the implementation of the main tasks.

The first main task for this project is comparison. Namely, we want to compare the path and performance of different
requests. Each one of our traces represents a request and the collection of events in one trace forms a directed acyclic graph. Our
comparison tasks are meant to compare the structure of the DAGs created by the events and the duration of different requests. 
We want to support three different comparison tasks: one trace against one trace, one trace against many traces, and many traces
against many traces. In more abstract terms, we will compare the DAGs formed by the events in different traces by partitioning them
into side-by-side views or by showing some sort of a graph diff.

The second major task we are proposing is summarizing data. Many of the traces in our datasets are similar, so we want to
aggregate traces with the same tags and events. We believe aggregating traces with the same tags or events will give the user
a more generalized understanding of the traces. The user will be able to analyze the average duration of a group of traces, instead
of relying on the data from one single trace. This task will take the DAGs formed by the events of different traces and will summarize
them by aggregating traces with the same tags or events, so that we can better understand the topology and paths of these graphs.

There are two more tasks we want to support, but may be out of scope for the project so we are leaving these tasks as stretch goals.
The first task is creating a dependency graph using
the processes in a trace and the second task is adding source code integration to our tool. The \textit{process\_name} attribute in 
our dataset gives the name of a service in a microservice.
Developers building distributed systems are interested in understanding the structure of the microservice
architecture they are building. We want to consume the list of events in a trace and use the \textit{process\_name} attribute in an 
event to build a graph that links processes that trigger other processes. This information will be presented to developers so that 
they can discover the dependencies that build their microservices. 
The second task consists of using the \textit{src\_line} and \textit{file\_path}
attributes to locate the line in the source code that triggered an event. We will do this by providing a hyperlink to the file and line
in the github repository. 