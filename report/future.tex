\section{Future Work}

Although, we believe that TraViz already provides better visualization
idioms for analysing distributed traces, TraViz is far from ready from
being deployed in a real world setting.

The main goal for our future work is to improve upon our graph comparison
and aggregation idioms so as to remove the "hairball effect". This requires
using better layout algorithms for visually laying out graphs on the screen.
We have identified potential libraries like cytoscape and d3-dag that provide
better layout algorithms but we haven't been able to get those to work yet.

Based on the feedback received from the target users, there are some minor
improvements that we would like to make in our existing visualization idioms.
Specifically, for the swimlane view of the trace, we would like to further
encode the ID of the process as the color of the swimlane for the thread.
This would help users identify which process a thread belongs to and
understand when the request crosses process boundaries. Additionally, we would
like to add a small detail window to the swimlane idiom that shows the details
of an event when the corresponding node is clicked on the main visualization.
We would also like to switch the dependency graph to be a directed graph
as that is what our target users are more accustomed to a directed dependency
graph.

Additionally, we want to design idioms for two more tasks. The first task
is the comparison between a single trace and an aggregation of traces
that would help users understand how a single, anomalous trace is
different from a set of normal looking traces. The second task is the comparison
between two different aggregations of traces. This would help understand
developers to understand the differences in between clusters of traces.

Lastly, we believe that conducting a formal user study or at least an informal
user study with more users would help in gauging the efficacy and usability
of TraViz as a distributed tracing analysis tool.