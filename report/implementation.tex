\section{Implementation Approach}

The Implementation of our solution consists of 3 parts: (i) MySQL database, (ii) Backend REST API
server implemented in Go, and (iii) Frontend React WebApp that implements our visualization idioms.

The MySQL database contains 4 different tables. The 4 tables are the overview, sourcecode, tags,
and dependencies table. These tables contain values for various derived attributes
that we make use of in our idioms. The database setup code is 39 lines of MySQL.

The Go Backend is responsible for processing the raw JSON format of distributed traces
and populating the tables created in the MySQL database. The backend also
hosts a REST API server to service requests from the frontend which queries over the database
to obtain information about the traces. The backend implementation also contains
a minimal graph kernel library for computing the similarities between 2 graphs.
Specifically, we implemented the Weisfeiler-Lehman graph kernel ~\cite{shervashidze2011weisfeiler}.
The Go Backend, including the graph kernel library, is implemented in 1600 lines of code.

The frontend is a React web application where we implement our visualization idioms.
We use d3 to implement our viz idioms. We make use of the dc and crossfilter libraries
to implement our overview and source code dashboard to provide linked views. We use
the react-d3-graph library to implement our graph-based visualization idioms.
Our swimlane visualization idiom for showing a detailed view of a single distributed trace
is based on the visualization idiom used by
X-Trace server ~\cite{fonseca2007x, xtraceserver} as we believe it is very effective
and very useful in providing detailed performance hints to the developers.
Our frontend implementation is 3000 lines of javascript and css on top of
the libraries we used.