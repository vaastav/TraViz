\documentclass[journal]{vgtc}
\usepackage[utf8]{inputenc}
\usepackage{geometry}
\usepackage{fullpage}
\usepackage{natbib}
\usepackage{adjustbox}
\usepackage{graphicx}
\usepackage{balance}
\usepackage{url}

\title{TraViz: Visualization of Traces in Distributed Systems}
\author{Vaastav Anand and Matheus Stolet}

\abstract{
    In this work we present TraViz, an interactive visualization tool for exploring and analysing 
    distributed traces to troubleshoot and debug performance problems in distributed systems.
    Through the use of composite filtering of multiple dimensions of the dataset, TraViz
    provides developers an easy to use way to find outlier traces. With TraViz's traceview
    swimlane idiom, the users can drill down into a single trace to analyze the trace
    for performance issues at the detail level of an operating system thread. TraViz's
    comparison idiom allows users to compare the Graphical structure of 2 traces of interest
    to highlight the key differences between the traces.
    The aggregation idiom allows users to find uncommon occuring events across traces by
    constructing a luminance coded super graph of all the distributed traces.
    Additionally, TraViz is the first visualization tool that provides an integrated view
    of the static source code of the system with dynamic information collected about the
    system via distributed traces which can help the users in identifying locations in the
    source code that would be ideal for performance optimizations.
}

\keywords{Distributed Tracing, Graph Comaprison, Graph Aggregation, Source Code Integration}

\teaser{
    \centering
    \includegraphics[width=\linewidth]{fig/crossfilter}
    \caption{TraViz overview layout, showing a filtered selection of distributed traces using filters on distributions of multiple dimensions of the distributed traces dataset.}
      \label{fig:teaser}
}

\date{}

\vgtcinsertpkg

\begin{document}

\maketitle

\section{Introduction}

Distributed systems are prevalent in society to the extent that billions of people either directly or 
indirectly depend on the correct functioning of a distributed system. From banking applications to social
networks, from large-scale data analytics to online video streaming, from web searches to cryptocurrencies,
most of the successful computing applications of today are powered by distributed systems.
The meteoric rise of cloud computing in the past decade has only increased our dependence on these
distributed systems in our lives.

Tasks like monitoring, root cause analysis, performance comprehension require techniques that cut across component,
system, and machine boundaries to collect, correlate, and integrate data. Distributed Tracing is one such cross-cutting technique
that correlates events across the system to a specific request by propagating a unique context per system with the request.
A trace represents the path of one request through the system and contains information such as the timing of requests, 
the events executed, and the nodes where these events were executed. Moreover, traces can be used
to identify slow requests and understand the difference between request executions. 

Although distributed traces carry vital information for debugging and for general system understanding,
there have been question marks against the usability of distributed tracing ~\cite{sridharandisttracing,kleindisttracing}. This has been primarily
due to the lack of good analysis tools for analysing the datasets. Specifically, existing visualization
tools don't provide an interface for exploring the dataset of traces and finding potential outliers.
Additionally the tools don't have a way of showing structural differences between traces and 
structural similarities across a group of traces. Distributed Traces are such a rich source
of data that can be useful for debugging and even resource allocation but existing tools
have limited the usability of distributed tracing as these tools have barely scratched the surface
of the plethora of analysis tasks that can be carried with the data available
in distributed traces.

To rectify the shortcomings of existing analysis tools and to make distributed tracing more useful,
in this paper, we present a new visualization tool called TraViz to analyze
and explore datasets of distributed traces. TraViz exposes an exploration dashboard which
allows the users to find outlier traces by filtering across distributions of multiple
dimensions of the trace dataset. TraViz also integrates the performance information with
the source code information of the distributed system, highlighting files and lines in the
source code that appear the most across traces. TraViz also provides idioms for comparing
two different traces as well as aggregating multiple traces into a single super-trace.
To provide the users with a sense of familiarity, TraViz also provides popular single-trace visualizations
that are prevalent in state-of-the-art distributed tracing visualization tools.
With the use of a tiny informal user study, we show that TraViz achieves our goals of improving
the usability of distributed tracing through enriching the analytical power of the users tasked
with analyzing the distributed trace datasets.

\section{Data}

We have a collection of traces and source code for a couple of different systems. Our traces come from two datasets.
The first dataset is called HDFS and contains the traces from a Haddop file system used for distributed storage 
and big data processing. This dataset has a total of 71,001 traces. The second dataset is called socialNetwork and
was obtained from the DeathStarBench open-source benchmark for cloud microservices. This dataset's traces model
the microservices of a social network composed of multiple individual applications. The socialNetwork dataset has
a total of 22,286 traces. The attributes of both datasets model five different entities: traces, events, hosts,
processes and threads. These entities are illustrated in \ref{fig:entities}.

\begin{figure}
    \includegraphics[width=\linewidth]{../data_abstractions.png}
    \caption{The different entities in TraViz.}
    \label{fig:entities}
  \end{figure}

\subsection{Trace}

A trace is a collection of events. It represents a request from a client to a service and shows the path of the
request through the microservice. A trace has three attributes: \textit{id}, \textit{list\_of\_tags}, and \textit{list\_of\_events}.
The \textit{id} of the trace is a categorical attribute used to identify a trace. The \textit{list\_of\_tags} is a list of human defined keywords that serve as
the metadata for the trace. There are on average two tags per trace in both datasets, but the socialNetwork
dataset has a total of 8 different tags and the HDFS dataset has a total of 22285 different tags. The
\textit{list\_of\_events} attribute is a list of the events that happened in a trace. The events in the list
are ordered, with events that caused another event preceding the caused event in the list. The causal relationship
between the events forms a DAG. There are around 100 events per trace in the socialNetwork dataset and 1400 events per trace
in the HDFS dataset.

\subsection{Event}

Events are important things that happen in a system, such as acquiring a lock, sending a request to another server,
or performing an update. In short, events are anything a developer things is useful enough to log. Events have 11
different attributes. They are \textit{id}, \textit{trace\_id}, \textit{thread\_id}, \textit{process\_id}, \textit{host\_name},
\textit{parent\_event\_ids}, \textit{hrt}, \textit{lrt}, \textit{src\_line}, \textit{file\_path}, and \textit{label}. The
\textit{id} is a categorical variable used to uniquely identify an event. The \textit{trace\_id} is another categorical
attribute that indentifies the trace that holds an event. The \textit{thread\_id} is a categorical variable that identifies the
thread that executed the event. The \textit{process\_id} gives the id of the process holding the thread that executed the event.
The \textit{host\_name} gives the name of the machine running the process that executed the event. An event can have at most one
host. The \textit{parent\_event\_ids} attribute is a collection of event ids. Each id maps to a parent event. The ids are used to
create a causal ordering between events. The socialNetwork and HDFS datasets have on average one parent event id per event, but 
can have up to two parent events for one event. 

The \textit{hrt} is a quantitative attribute that stands for high resolution time
stamp, and provides nanosecond level precision for the time an event was initiated. The \textit{lrt} is also a quantitative attribute.
It provides a milisecond level precision low resolution time stamp for the time an event occured. The \textit{src\_line} gives the line
in the source code where an event was logged. The \textit{file\_path} is a categorical attribute that gives the path to the file where 
a programmer logged this specific event. This attribute can be used with the \textit{src\_line} to find the exact line of code of a logged event.
The last attribute in an event is the \textit{label}. The label is a free-form text annotaion that gives a high-level description of the
event. Labels are a meaningful human added annotation for debugging, and there are 2,472 labels in the socialNetwork dataset and 210,957 labels
in the HDFS dataset.

\subsection{Host}

A host represents a node in the distributed system. Hosts encapsulate the processes and threads
that execute the events in a trace. There are 13 different hosts across the social network dataset
and 9 different hosts across the HDFS dataset. The host is identified by a categorical \textit{name} attribute , which is
referenced by events and threads.

\subsection{Process}

The process entity represents a operating system process running on
a host. Processes have a categorical \textit{id} attribute, which is referenced by the events and threads executed in
the process. The process also has a categorical \textit{name} attribute, which gives a human understandable name for the
application running the process. There are around 20 different processes per trace in the socialNetwork
dataset and 18 different processes per trace in the HDFS dataset. There are 13 different process names
in the socialNetwork dataset and 4 different process names in the HDFS dataset.

\subsection{Thread}

The thread entity represents
a kernel or user thread. Distinguishing between both is not important for the purposes of this project, so we
simply see it as the basic unit of execution. Threads have a categorical \textit{id} attribute that identifies a thread
within a trace. Events use the \textit{id} to signal the thread that executed them. It is important to note
that the thread \textit{id} does not match across traces, and the total number of threads is not useful because it's
not possible to correlate them between traces. Each thread also has two more categorical attributes: the \textit{host\_name}
and \textit{process\_id}. These attributes are used to identify the host and process that executed a thread. 

\section{Task}

For this project we want to focus on comparison tasks. Namely, we want to compare the path and performance of different
requests. Each one of our traces represents a request and the collection of events in one trace forms a directed acyclic graph. Our
comparison tasks are meant to compare the structure of the DAGs created by the events and the duration of different requests. 
We want to support three different comparison tasks: one trace against one trace, one trace against many traces, and many traces
against many traces. In more abstract terms, we will compare the DAGs formed by the events in different traces.  

\section{Scenario}

\begin{itemize}
\item \textbf{Scenario 1}: A developer wants to find out why two similar requests have very different 
completion times. The developer will select the two traces corresponding to the requests. After the traces are selected, our tool will
use the events of these traces to generate a view showing the differences between the traces as well as showing the
context and timing of the original requests. This will allow the developer to analyze and identify why 2 similar requests have 
different completion times.

\item \textbf{Scenario 2}: A developer wants to find out why requests on Monday are
slower than the requests on Tuesday. The developer will make 2 different selections - selection of traces from Monday and selection
of traces from Tuesday. Our tool will aggregate the 2 selections into representative graphs and then show the difference of these
two aggregate traces. This will allow the developer to possibly figure out a high-level change between the request execution from
Monday to Tuesday.

\item \textbf{Scenario 3}: A developer wants to analyze why a given trace is anomalous as compared to some of the previous
traces. To do this, the developer will first create a selection of traces that will be aggregated down into 1 trace. The developer
will then select the anomalous trace and create a comparison between the anomalous and the representative trace.
Our tool will show the difference between the 2 traces and allow the developer to figure out why a particular trace is anomalous.

\item \textbf{Scenario 4}: A developer wants to understand the communication load between different services of the system.
Our tool will show the developer an overview graph that shows how often 2 services in the distributed system communicate
with each other. This will allow the developer to figure out how an addition of a new service would increase the load on each
service.

\end{itemize}

\section{Proposed Solution}

The events of a trace will be transformed into a node-link graph. One possible idiom for trace comparison, is to compare the graphs formed by the events of different traces
by partitioning them into side-by-side views. Another possible idiom is showing the diff of the 2 graphs whilst still preserving
the original graphs to provide some sort of context.
Because of the limited screen space of a computer monitor, we will allow at most two
graphs to be compared at one time. Even though our tool will only allow two graphs to be compared at once, users will still be able
to compare multiple traces by aggregating traces that go through the exact same nodes into one graph. In other words, traces that have
the same structure can be aggregated into one graph. This aggregation procedure will allow us to make comparisons that involve many
traces while only using two DAGs. The links of the graph will be saturated according to the average time it takes for the event in
the end of the link to complete. We will also use colour hue to identify the \textit{process\_name} of each event. A preliminary sketch of
this visualization can be seen in Figure \ref{fig:comparison}.

We will use a similar approach to create the service dependency graph. To create the dependency graph for one trace, we will transform the list
of events into a DAG represented by a node-link graph. All the events with the same \textit{process\_name} will be aggregated into the same
node. In this representation, the aggreated node will represent a service in a microservice. The links to the node will be saturated according
to the amount of time it takes for all the events in a service to complete. A preliminary sketch of this visualization can be seen in Figure \ref{fig:aggregation}.

At the moment, our proposed solution for source code integration does not have a specific visualization in mind. We are thinking about
providing a link to the file and line in the github repo corresponding to the selected event in a trace. At least initially, we are not
going to do any complicated visualization to accomplish the source code integration task.

\begin{figure}
    \includegraphics[width=\linewidth]{comparison.jpg}
    \caption{Example of a comparison task where we compare one task that calls Postgres and another that calls Redis. The trace that
    called Postgres triggers slow events (represented by the red links) and triggers more events in general.}
    \label{fig:comparison}
\end{figure}

\begin{figure}
    \includegraphics[width=\linewidth]{agg.jpg}
    \caption{Example of an aggregation task where events with the same \textit{process\_name} get aggregated into the same node.}
    \label{fig:aggregation}
\end{figure}


\section{Implementation Approach}

Our visualization tool will be a webapp. We will host a backend written in Go that 
will serve requests from a client. The frontend will be written in javascript. We will use D3 to manipulate the data served
by our backend and to create the visualizations necessary to fulfill our comparison tasks.

\section{Related Work}

\textbf{TODO: Add Related}

\balance
{\footnotesize
\bibliographystyle{abbrv}
\bibliography{report}
}

\end{document}

