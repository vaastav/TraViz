\section{Introduction}

Distributed systems are prevalent in society to the extent that billions of people either directly or 
indirectly depend on the correct functioning of a distributed system. From banking applications to social
networks, from large-scale data analytics to online video streaming, from web searches to cryptocurrencies,
most of the successful computing applications of today are powered by distributed systems.
The meteoric rise of cloud computing in the past decade has only increased our dependence on these
distributed systems in our lives.

Tasks like monitoring, root cause analysis, performance comprehension require techniques that cut across component,
system, and machine boundaries to collect, correlate, and integrate data. Distributed Tracing is one such cross-cutting technique
that correlates events across the system to a specific request by propagating a unique context per system with the request.
A trace represents the path of one request through the system and contains information such as the timing of requests, 
the events executed, and the nodes where these events were executed. Moreover, traces can be used
to identify slow requests and understand the difference between request executions. 

Although distributed traces carry vital information for debugging and for general system understanding,
there have been question marks against the usability of distributed tracing ~\cite{sridharandisttracing,kleindisttracing}. This has been primarily
due to the lack of good analysis tools for analysing the datasets. Specifically, existing visualization
tools don't provide an interface for exploring the dataset of traces and finding potential outliers.
Additionally the tools don't have a way of showing structural differences between traces and 
structural similarities across a group of traces. Distributed Traces are such a rich source
of data that can be useful for debugging and even resource allocation but existing tools
have limited the usability of distributed tracing as these tools have barely scratched the surface
of the plethora of analysis tasks that can be carried with the data available
in distributed traces.

To rectify the shortcomings of existing analysis tools and to make distributed tracing more useful,
in this paper, we present a new visualization tool called TraViz to analyze
and explore datasets of distributed traces. TraViz exposes an exploration dashboard which
allows the users to find outlier traces by filtering across distributions of multiple
dimensions of the trace dataset. TraViz also integrates the performance information with
the source code information of the distributed system, highlighting files and lines in the
source code that appear the most across traces. TraViz also provides idioms for comparing
two different traces as well as aggregating multiple traces into a single super-trace.
To provide the users with a sense of familiarity, TraViz also provides popular single-trace visualizations
that are prevalent in state-of-the-art distributed tracing visualization tools.
With the use of a tiny informal user study, we show that TraViz achieves our goals of improving
the usability of distributed tracing through enriching the analytical power of the users tasked
with analyzing the distributed trace datasets.