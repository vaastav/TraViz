\section{Conclusion}

In this paper we presented TraViz, a tool designed
for analyzing the structural and performance attributes
of distributed traces. TraViz supports 6 different analysis
tasks - detection of outlier traces, single trace performance analysis,
source code integration with performance metrics for future code optimizations,
comparison of 2 traces to detect differences, aggregation of traces to
highlight common structure and find outlier events, and dependency analysis
between different services in a distributed system.
There has been prior research in visualizing traces but most of it has been
focused on performing single trace performance analysis. TraViz extends
this by allowing users to explore different traces to find interesting traces
and then perform comparison and/or aggregation actions on these traces.
As far as we know, TraViz is the first visualization tool that implements
a visualization for combining static source code information with
dynamic information about the system collected via traces. 
That being said, we feel that TraViz's graph visualizations are still not
useful to developers in its current state and require further refinement
before they can be perceived useful.